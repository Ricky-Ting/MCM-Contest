为了考虑相邻县之间的联系,我们引入元胞自动机模型。但我们对基本的元胞自动机模型做了很多改进来适应我们的问题。

	To consider the relationship between adjacent 
counties, CA model is introduced. Due to the specificity of this problem, we will modify the traditional CA model to satisfy the limitations of this problem. \par

	In our revised CA model, a cell represents some county in the five states and the state of a cell denotes the number of predicted drug reports of some county. Any two cells are adjecent in CA model if and only if the two counties they respectively represent are geographically adjecnt. Traditional CA model has to restrict the number of neighbours to be 4 or 8. However, we shed the constraint (or break the limit). In our model, the number of neighbours can be variable. \par

	Variable:
		$c_{j}$
		$s_{ij}$
		$n_{j}$
		$adj[j]$


	Rules:\\
	\subsection{Prediction Rules }
		Any cell's state in the $(i+1)_{th}$ stage is determined by its state and its neighbours' state in the $i_{th}$ stage. In other words, $s_{(i+1)j}$ is determined by $s_{ij}$ and $s_{ik}$ where $c_k$ is the neighour of $c_j$.\par
		With careful consideration, We choose the relationship to be linear. The Prediction formula is as follows.
			$$ s_{(i+1)j} = k_1 s_{ij} + k_2 \lambda \frac{\sum _{k=1} ^{n_j} s_{i adj[j][k]} }{n_j} $$. (Prediction model)
		where $k_1$ denotes the influence of its own state, $k_2$ denotes the infulence of its neighbours' state, $\lambda$ is a constant which describes the degree of adjecnt counties's contact. After multiple tests and the consideration of reality, we choose $\lamba$ to be 0.3.\par
	\subsection{Backstepping Rules}
		We also need to estimate the data in years before 2010 which is not provided. We simply rewrite the formulation above and make some approximation. To distinguish Prediction formula, we use $s_{ij}'$ to replace $s_{ij}$. The formula is as follows:
			$$ s_{ij}' = \frac{s_{(i+1)j}' - k_2 \lambda \frac{\sum _{k=1} ^{n_j} s_{(i+1) adj[j][k]}' }{n_j} }{k_1} $$
			where $k_1$ , $k_2$ and $\lambda$ are identical to those in Prediction model. 

		\subsection{Determine coefficients} \label{Section:Determine coefficients}
		To determine the coefficients $k_1$ and $k_2$, we use program to help us choose between $0$ and $1$ automatically. We have data about drug reports in years 2010-2017. The program will choose between $0$ and $1$ by step $0.001$, to minimize the error. The formula to calculate error is as follows.
			$$error = \frac{\sum_{i=2010}^{2017} \sum_{j=1}^{N} (r_{ij} - s_{ij})^2  }{N * (2017-2010+1)}$$
		where $N$ is the number of counties in the five state.


		\subsection{Results}
			We have estimated the number of drug reports of every county in years from 2000 to 2030 using our revised CA model. Therefore we can estimate the data of every state by summing the data of its counties. Fig.2 shows the estimated and actual data of Pennsylvania. we can balabala 看图说话。 And Fig.3 shows the   of Virginia, 看图说话。  分析一通

			With estimated data of every county, we are able to describe the geographical distribution. In Fig.5, we can see 







